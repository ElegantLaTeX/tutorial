\documentclass[8pt]{beamer}

\usefonttheme{serif}
% \geometry{paperwidth=140mm,paperheight=105mm}

\RequirePackage[no-math]{fontspec}
\setmainfont{texgyretermes}[
  UprightFont = *-regular ,
  BoldFont = *-bold ,
  ItalicFont = *-italic ,
  BoldItalicFont = *-bolditalic ,
  Extension = .otf ,
  Scale = 1.0]
  
\setsansfont{texgyreheros}[
  UprightFont = *-regular ,
  BoldFont = *-bold ,
  ItalicFont = *-italic ,
  BoldItalicFont = *-bolditalic ,
  Extension = .otf ,
  Scale = 0.9]

\RequirePackage[UTF8,scheme=plain,fontset=none]{ctex}
\setCJKmainfont[BoldFont={FZHei-B01},ItalicFont={FZKai-Z03}]{FZShuSong-Z01}
\setCJKsansfont[BoldFont={FZHei-B01}]{FZKai-Z03}
\setCJKmonofont[BoldFont={FZHei-B01}]{FZFangSong-Z02}
\setCJKfamilyfont{zhsong}{FZShuSong-Z01}
\setCJKfamilyfont{zhhei}{FZHei-B01}
\setCJKfamilyfont{zhkai}[BoldFont={FZHei-B01}]{FZKai-Z03}
\setCJKfamilyfont{zhfs}[BoldFont={FZHei-B01}]{FZFangSong-Z02}
\newcommand*{\songti}{\CJKfamily{zhsong}}
\newcommand*{\heiti}{\CJKfamily{zhhei}}
\newcommand*{\kaishu}{\CJKfamily{zhkai}}
\newcommand*{\fangsong}{\CJKfamily{zhfs}}

\usepackage{anyfontsize}
\usepackage[english]{babel}
\usepackage{scrextend}
\addtokomafont{labelinglabel}{\normalfont\bfseries}
\usepackage{amsmath,amssymb}
\usepackage{bookmark}

\usetheme[sectionpage=none]{metropolis}
\setlength\abovecaptionskip{-1pt}
\setlength\belowcaptionskip{-1pt}
% adjust the space between caption and figure and table
% source: https://tex.stackexchange.com/questions/197879/how-to-adjust-spacing-between-caption-and-table-figure-in-beamer
\usefonttheme{professionalfonts}
\titlegraphic{\includegraphics[width=1cm]{logo.png}}


\setbeamerfont{normal text}{family=\kaishu\sffamily}
\setbeamerfont{frametitle}{family=\large\bfseries\sffamily}
\setbeamerfont{title}{family=\bfseries}
\setbeamerfont{subtitle}{family=\kaishu}
\setbeamerfont{institute}{size=\small}
\definecolor{iron}{RGB}{0,82,67}
\setbeamercolor{frametitle}{bg=iron}
\setbeamercolor{progress bar}{fg=iron,bg=iron}


\setbeamertemplate{section in toc}[ball unnumbered]
\setbeamerfont{section in toc}{family=\heiti\sffamily}
\setbeamerfont{myTOC}{}
\AtBeginSection[]{\frame{\frametitle{目录}%
                  \usebeamerfont{myTOC}\tableofcontents[current]}}

\setcounter{tocdepth}{1}
\makeatletter
\setbeamertemplate{frametitle}[default][left,leftskip=\the\dimexpr\beamer@leftmargin-0.6cm\relax]
\makeatother


\setbeamertemplate{footline}
{%
   \leavevmode%
 \hskip1cm \fangsong \insertshortauthor \quad|\quad \insertshorttitle \hfill \normalfont  \insertframenumber\,/\,\inserttotalframenumber  \hskip1cm \hfill
   \vskip6pt%
}

\graphicspath{{image/}{figure/}{fig/}{img/}}
\usepackage{booktabs}
\usepackage{silence}
\WarningFilter{latexfont}{Font shape}
\let\scshape\relax
\usepackage{type1cm}
\AtBeginDocument{\usebeamerfont{normal text}}

\newcommand\email[1]{\href{mailto:#1}{\nolinkurl{#1}}}

\renewcommand\ttdefault{lmtt}

\author{ddswhu, syvshc, sikouhjw, osbertwang}
\title{Elegant\LaTeX{} 分享会}
\subtitle{模板使用介绍}
\institute{ElegantLaTeX Program}
\date{2022/04/23}


\titlegraphic{%
  \begin{picture}(0,0)
    \put(50, -5){\makebox(0,0)[rt]{\includegraphics[width=1.2cm]{logo.png}}}
    \put(280,-150){\makebox(0,0)[rt]{\includegraphics[width=2cm]{logo-blue.png}}}
  \end{picture}}

\begin{document}

\maketitle


\begin{frame}\frametitle{目录}
\tableofcontents[hideallsubsections]

\end{frame}


\section{介绍}

\begin{frame}{关于 ElegantLaTeX}
Elegant\LaTeX{} 项目组由 \textcolor{iron}{ddswhu} 和\textcolor{iron}{小 L} 成立于 2012 年,致力于打造一系列美观、优雅、简便的模板方便用户使用。目前由 \href{https://github.com/ElegantLaTeX/ElegantNote}{ElegantNote},\href{https://github.com/ElegantLaTeX/ElegantBook}{ElegantBook},\href{https://github.com/ElegantLaTeX/ElegantPaper}{ElegantPaper} 组成,分别用于排版笔记,书籍和工作论文。

我们的联系方式如下,建议加入用户 QQ 群提问,这样能更快获得准确的反馈,加群时请备注 \LaTeX{} 或者 Elegant\LaTeX{} 相关内容。
\begin{itemize}
  \item 官网:\href{https://elegantlatex.org/}{https://elegantlatex.org/}
  \item GitHub 地址:\href{https://github.com/ElegantLaTeX/}{https://github.com/ElegantLaTeX/}
  \item Gitee 地址:\href{https://gitee.com/ElegantLaTeX}{https://gitee.com/ElegantLaTeX}
  \item CTAN 地址:\href{https://ctan.org/pkg/elegantbook}{https://ctan.org/pkg/elegantbook},note 和 paper 类似。
  \item 微博:Elegant\LaTeX{}
  \item 微信公众号:Elegant\LaTeX{}
  \item 用户 QQ 群:692108391
  \item 邮件:\email{elegantlatex2e@gmail.com}
\end{itemize}
\end{frame}

\begin{frame}{人员介绍}
  \begin{itemize}
    \item \textcolor{iron}{\bfseries 邓东升}:复旦大学经济学博士,又名 \underline{ddswhu}、\underline{EthanDeng},Elegant\LaTeX{} 的创立者和维护者。2010 年开始接触使用 \LaTeX{}。国内最早 Sublime Text 和 VS Code 推广者之一。\\[2ex]
    \item \textcolor{iron}{\bfseries 王然}:中国科学院大学博士后,又名\underline{啸行},\underline{ObsertWang}。目前负责维护 install-latex-guide-zh-cn 手册,曾参加过 Overleaf 公司 Advisor 项目,对多种编辑器、版本和 Overleaf 非常熟悉。
    \item \textcolor{iron}{\bfseries 孙忠豪}:中科院数学与系统研究院硕士,又名\underline{乙醇},\underline{syvshc},Elegant\LaTeX{} 成员,HITBeamer 模板(哈工大)开发者和维护者,活跃在 QQ 用户群和 Github,校对翻译过 \TeX{} Live 包管理器的文档,热爱分享。\\[2ex]
    \item \textcolor{iron}{\bfseries 何骏炜}:广东工业大学硕士,又名\underline{死抠},\underline{sikouhjw},Elegant\LaTeX{} 成员,gdutthesis 开发者和维护者。自 2019 年开始在用户群群活跃,解答了 Github/知乎/LaTeXStudio.net/Stack Overflow 上诸多关于本模板的问题。\\[2ex]
  \end{itemize}
\end{frame}


\begin{frame}{使用须知-适合人群}
  \begin{itemize}
    \item 对 \LaTeX{} 有一定了解,能够正常使用;
    \item 不习惯用默认的 article、book 文类;
    \item 文稿中有较多的数学公式;
    \item 接受模板的大部分设定;
    \item 认可模板的审美。
  \end{itemize}  
\end{frame}


\begin{frame}{使用须知-不适合人群}
  \begin{itemize}
    \item 无法自行编译文档;
    \item 自定义想法很多;
    \item 具备单独写模板的能力;;
    \item 完全不想折腾,常年未更新过 TeX Live;
    \item CTeX 用户。
  \end{itemize}  
\end{frame}

\section{Elegant\LaTeX{} 模板:本地与在线使用}

\begin{frame}{本地使用:\TeX{} Live 2022 的安装}
  演示。
\end{frame}

\begin{frame}{在线使用:Overleaf 在线使用}
  演示。
\end{frame}

% !TeX root = main.tex
\section{使用命令行编译}

\subsection{调出命令行工具}

\begin{frame}\frametitle{打开终端}
  \begin{description}
    \item[Windows 用户] 
    \begin{itemize}
      \item \keys{\win + S} 后输入 ``cmd'' 并按下回车, 可以打开``命令提示符'', 也就是所谓的命令行, 或者使用 \keys{\ctrl + \shift + \enter} 来用管理员权限打开命令行.
      \item 在``文件资源管理器''的地址栏输入``cmd'' 并按下回车, 即可在当前位置打开命令行:
      \item 如果安装了 Windows Terminal, 或者系统为 Win11, 可以在文件夹中或者文件夹上点击右键, 选择``在终端中打开'', 即可打开 Windows Terminal:
    \end{itemize}
    \item[macOS 用户] \keys{\cmdmac + \SPACE} 后输入``terminal'', 即可打开终端. 
    \item[Linux 用户] \keys{\ctrl + \Alt + T} 即可打开终端. 
  \end{description}
后面我们提到终端, 命令行均指这些东西, 并不做名字上的区分
\end{frame}


\begin{frame}[fragile]
  \frametitle{命令行操作}
  \begin{enumerate}
    \item \cmd{cd};
    \item \cmd{dir} (Windows), \cmd{ls} (Linux, macOS);
    \item \cmd{code}
  \end{enumerate}
\end{frame}


\begin{frame}[fragile]{使用命令行编译 \LaTeX 文件}
  \framesubtitle{第一次编译}
  \begin{enumerate}
    \item 在 \directory{D:/folder} 文件夹下新建 \directory{main.tex}, 输入以下内容
      \begin{latexcode}
      \documentclass{article}
      \begin{document}
        Hello \LaTeX!
      \end{document}  
      \end{latexcode}
    \item 在命令行中运行 
\begin{cmdcode}
pdflatex main.tex
\end{cmdcode}
      \begin{outputcode}
      This is pdfTeX, Version 3.141592653-2.6-1.40.24 (TeX Live 2022) (preloaded format=pdflatex)
      restricted \write18 enabled.
      entering extended mode
      (./main.tex
      LaTeX2e <2021-11-15> patch level 1
      L3 programming layer <2022-02-24>
      (/home/syvshclily/texlive/2022/texmf-dist/tex/latex/base/article.cls
      Document Class: article 2021/10/04 v1.4n Standard LaTeX document class
      (/home/syvshclily/texlive/2022/texmf-dist/tex/latex/base/size10.clo))
      (/home/syvshclily/texlive/2022/texmf-dist/tex/latex/l3backend/l3backend-pdftex.
      def) (./main.aux) [1{/home/syvshclily/texlive/2022/texmf-var/fonts/map/pdftex/u
      pdmap/pdftex.map}] (./main.aux) )</home/syvshclily/texlive/2022/texmf-dist/font
      s/type1/public/amsfonts/cm/cmr10.pfb>
      Output written on main.pdf (1 page, 12962 bytes).
      Transcript written on main.log. 
      \end{outputcode}
  \end{enumerate} 
\end{frame}

\subsubsection{遇到错误}

\begin{frame}[fragile]{使用命令行编译 \LaTeX 文件}
\framesubtitle{遇到错误}
  \begin{outputcode}
  This is pdfTeX, Version 3.141592653-2.6-1.40.24 (TeX Live 2022) (preloaded format=pdflatex)
  restricted \write18 enabled.
  entering extended mode
  (./main.tex
  LaTeX2e <2021-11-15> patch level 1
  L3 programming layer <2022-02-24>
  (/home/syvshclily/texlive/2022/texmf-dist/tex/latex/base/article.cls
  Document Class: article 2021/10/04 v1.4n Standard LaTeX document class
  (/home/syvshclily/texlive/2022/texmf-dist/tex/latex/base/size10.clo))
  (/home/syvshclily/texlive/2022/texmf-dist/tex/latex/l3backend/l3backend-pdftex.
  def) (./main.aux)
  ! Undefined control sequence.
  l.3   Hello \Latex
                    !
  ?  
  \end{outputcode}

  \begin{itemize}
    \item \keys{I + <cs> + \enter};
    \item \keys{X + \enter};
    \item \keys{H + \enter}.
  \end{itemize}
\end{frame}

\subsubsection{编译选项}

\begin{frame}[fragile]
  \frametitle{使用命令行编译 \LaTeX 文件}
  \framesubtitle{编译选项}

\begin{cmdcode}
pdflatex --option1 --option2=<string> main.tex
\end{cmdcode}
  
\begin{enumerate}
  \item \cmd{--synctex=1}
  \item \cmd{--jobname=<string>}
  \item \cmd{--output-directory=<string>}
  \item \cmd{--shell-escape}
  \item \cmd{--halt-on-error}
  \item \cmd{--interactionmode=<errorstop|scroll|batch|nonstop>mode}
\end{enumerate}
\end{frame}

\subsubsection{多次编译}

\subsubsection{多次编译}

\begin{frame}[fragile,t]
  \frametitle{使用命令行编译 \LaTeX 文件}
  \framesubtitle{目录与交叉引用}
\begin{latexcode}
% main.tex
\documentclass{article}
\begin{document}
\tableofcontents
\section{One}\label{sec:one}
  We are in section \ref{sec:one}
\end{document}
\end{latexcode}

\begin{onlyenv}<2>
\begin{cmdcode}
pdflatex main.tex
pdflatex main.tex
\end{cmdcode}  
\end{onlyenv}

\only<3>{
  \begin{center}
    \includegraphics[height=3cm]{cross-ref-right.png}
  \end{center}
}
\end{frame}


\begin{frame}[fragile, t]
  \frametitle{使用命令行编译 \LaTeX 文件}
  \framesubtitle<1-3>{参考文献 --- \texttt{bibtex}}

\begin{onlyenv}<1-3>
\begin{latexcode}
\documentclass{article}
\begin{document}
  text\cite{article-full}
  \bibliographystyle{plain}
  \bibliography{xampl.bib}
\end{document}
\end{latexcode}
\end{onlyenv}

\begin{onlyenv}<2>
\begin{cmdcode}
pdflatex main.tex
bibtex main.aux
pdflatex main.tex
pdflatex main.tex
\end{cmdcode}  
\end{onlyenv}

\begin{onlyenv}<3>
  \begin{center}
    \includegraphics[width=\textwidth]{bibtex-ref.png}
  \end{center}
\end{onlyenv}

\framesubtitle<4-6>{参考文献 --- \texttt{biber}}

\begin{onlyenv}<4-6>
\begin{latexcode}
\documentclass{article}
\usepackage{biblatex}
\addbibresource{xampl.bib}
\begin{document}
  text\cite{article-full}
  \printbibliography
\end{document}
\end{latexcode}
\end{onlyenv}

\begin{onlyenv}<5>
  \begin{cmdcode}
  pdflatex main.tex
  biber main.bcf
  pdflatex main.tex
  pdflatex main.tex
  \end{cmdcode}  
  \end{onlyenv}
  
  \begin{onlyenv}<6>
    \begin{center}
      \includegraphics[width=\textwidth]{biber-ref.png}
    \end{center}
  \end{onlyenv}

\end{frame}

\subsubsection{更强大的工具: \texttt{latexmk}}

\begin{frame}[fragile]
  \frametitle{使用命令行编译 \LaTeX 文件}
  \framesubtitle{更强大的工具: \texttt{latexmk}}

\begin{onlyenv}<1>
\begin{latexcode}
% main.tex
\documentclass{article}
\begin{document}
\tableofcontents
\section{One}\label{sec:one}
  We are in section \ref{sec:one}, 
  and we have a cite \cite{article-full}
  \bibliographystyle{plain}
  \bibliography{xampl.bib}
\end{document}
\end{latexcode}

\begin{cmdcode}
latexmk -pdf main.tex
\end{cmdcode}
\end{onlyenv}

\begin{onlyenv}<2>  
\begin{latexcode}
% main.tex
\documentclass{ctexart}
\usepackage[style=gb7714-2015]{biblatex}
\addbibresource{xampl.bib}
\begin{document}
\tableofcontents
\section{测试节}\label{sec:one}
  我们现在是第 \ref{sec:one} 节,
  我们有一个引用 \cite{article-full}
  \printbibliography
\end{document}
\end{latexcode}

\begin{cmdcode}
latexmk -xelatex main.tex
\end{cmdcode}
\end{onlyenv}

\end{frame}

\subsubsection{更多}

\begin{frame}{使用命令行编译 \LaTeX 文件}
  \framesubtitle{更多的内容}
  latexmk 命令文件
  \begin{itemize}
    \item \cmd{latexmk -c}, \cmd{latexmk -C};
    \item \texttt{latexmkrc} 文件.
  \end{itemize}
  
  更多内容可以参见
  \begin{itemize}
    \item \texttt{\textbf{texdoc} latexmk}
    \item \cmd{pdflatex --help}
    \item \href{https://syvshc.github.io/2022-03-06-latex-terminal-compiling/}{在终端中编译 \LaTeX}
  \end{itemize}
\end{frame}

\section{使用 \tlmgr 进行包管理}

\begin{frame}[fragile]{\tlmgr 命令介绍}
\begin{itemize}
  \item \tlmgr 是 \texlive 用来管理\emph{软件包}的工具;
  \item \emph{软件包}指的不只是可以通过 \lcmd{\usepackage} 调用的宏包, 而是所有 \texlive 包含的, 可以使用 \tlmgr 管理的内容, 比如平常所说的宏包, 如 \lstinline{amsmath.sty}, 一些说明文档, 如 \lstinline{lshort-zh-cn}, 一些可执行文件, 如 \lstinline{xetex.exe} 等等.
  \item \tlmgr 是 \textbf{T}eX\textbf{L}ive \textbf{M}ana\textbf{g}e\textbf{r} 的缩写;
  \item 命令行中运行 \cmd{tlmgr version} 就可以看到当前的\emph{修订版号}(revision), 通常输出如下
\begin{outputcode}
tlmgr revision 63033 (2022-04-15 07:19:42 +0200)
tlmgr using installation: C:/texlive/2022
TeX Live (https://tug.org/texlive) version 2022
\end{outputcode}
\end{itemize}
\end{frame}

\subsection{\texttt{tlmgr} 的基本语法}

\begin{frame}[fragile]{\tlmgr 的基本语法}
\begin{cmdcode}
tlmgr -global-options action -action-specific-options operand
\end{cmdcode}
\begin{enumerate}
  \item \cmd{-} 与 \cmd{--} 相同;
  \item 顺序不重要;
  \item \cmd{-dry-run}.
\end{enumerate}
\end{frame}

\subsection{查看宏包信息}

\begin{frame}[fragile]{查看宏包说明信息 --- 操作: \frameaction{info}}
\begin{cmdcode}
tlmgr info pkg1 pkg2
\end{cmdcode}
  \begin{outputcode}
package:     elegantbook
category:    Package
shortdesc:   An Elegant LaTeX Template for Books
longdesc:    ElegantBook is designed for writing Books. This template is based on the standard LaTeX book class. The goal of this template is to make the writing process more elegant. Just enjoy it!
installed:   Yes
revision:    62989
sizes:       doc: 2613k, run: 49k
relocatable: No
cat-version: 4.3
cat-license: lppl1.3c
cat-topics:  class chinese book-pub
cat-contact-home: https://elegantlatex.org/
cat-contact-announce: https://elegantlatex.org/
cat-contact-repository: https://github.com/ElegantLaTeX/ElegantBook
cat-contact-support: https://github.com/ElegantLaTeX/ElegantBook/issues
collection:  collection-latexextra
  \end{outputcode}
  \end{frame}

\begin{frame}[fragile]{查看宏包说明信息 --- 操作: \frameaction{info}}

\begin{cmdcode}
tlmgr info -list pkg1 pkg2
\end{cmdcode}
\begin{outputcode}
# 省略 tlmgr info elegantbook 的内容 #
Included files, by type:
run files:
  texmf-dist/tex/latex/elegantbook/elegantbook.cls
doc files:
  texmf-dist/doc/latex/elegantbook/License
  texmf-dist/doc/latex/elegantbook/README-CN.md
  texmf-dist/doc/latex/elegantbook/README.md details="Readme"
  texmf-dist/doc/latex/elegantbook/elegantbook-cn.pdf details="Package documentation (Chinese)" language="zh"
  texmf-dist/doc/latex/elegantbook/elegantbook-cn.tex
  texmf-dist/doc/latex/elegantbook/elegantbook-en.pdf details="Package documentation (English)"
  texmf-dist/doc/latex/elegantbook/elegantbook-en.tex
  texmf-dist/doc/latex/elegantbook/figure/cover.jpg
  texmf-dist/doc/latex/elegantbook/figure/logo-blue.png
  texmf-dist/doc/latex/elegantbook/image/founder.png
  texmf-dist/doc/latex/elegantbook/image/scatter.jpg
  texmf-dist/doc/latex/elegantbook/reference.bib
\end{outputcode}
\end{frame}


\begin{frame}[fragile]{修改宏包更新源 --- 操作: \frameaction{option}}

\begin{cmdcode}
tlmgr option repository mirror
\end{cmdcode}

值得一提: \cmd{repository} = \cmd{repo}
\end{frame}

\begin{frame}[fragile]{更新软件包 --- 操作: \frameaction{update}}
\framesubtitle{查看哪些软件包可以更新}

\begin{cmdcode}
tlmgr update -list
\end{cmdcode}
\begin{outputcode}
tlmgr.pl: package repository https://mirrors.tuna.tsinghua.edu.cn/CTAN/systems/texlive/tlnet (not verified: gpg unavailable)
tlmgr.pl: would save backups to C:/texlive/2022/tlpkg/backups
tlmgr.pl: skipping forcibly removed package: collection-texworks
update:   adjmulticol   [316k]: local:  62935, source:  63073
update:   hitex         [2565k]: local:  62529, source:  63073
update:   texlive-fr    [1394k]: local:  62853, source:  63071
update:   texlive-msg-translations [144k]: local:  63010, source:  63072
update:   texlive-scripts.win32 [36k]: local:  62199, source:  63068
update:   texlive-scripts [504k]: local:  63049, source:  63068
update:   utfsym         [4766k]: local:  56729, source:  63076
update:   xduts          [871k]: local:  63013, source:  63075
update:   zwpagelayout   [641k]: local:  53965, source:  63074  
\end{outputcode}
\end{frame}

\begin{frame}[fragile]{更新软件包 --- 操作: \frameaction{update}}
\framesubtitle{更新全部软件包}

\begin{cmdcode}
tlmgr update -self -all
\end{cmdcode}
\begin{outputcode}
tlmgr.pl: package repository https://mirrors.tuna.tsinghua.edu.cn/CTAN/systems/texlive/tlnet (not verified: gpg unavailable)
tlmgr.pl: saving backups to C:/texlive/2022/tlpkg/backups
tlmgr.pl: no self-updates for tlmgr available
[ 1/16, ??:??/??:??] update: babel-french [567k] (59997 -> 63088) ... done
[ 2/16, 00:02/01:48] update: l3backend [894k] (63025 -> 63089) ... done
...
[16/16, 01:23/01:23] update: collection-luatex [1k] (62829 -> 63081) ... done
running mktexlsr ...
done running mktexlsr.
running mtxrun --generate ...
done running mtxrun --generate.
running updmap-sys ...
...
tlmgr.pl: package log updated: C:/texlive/2022/texmf-var/web2c/tlmgr.log
tlmgr.pl: command log updated: C:/texlive/2022/texmf-var/web2c/tlmgr-commands.log
\end{outputcode}
\end{frame}

\begin{frame}[fragile]{更新软件包 --- 操作: \frameaction{update}}
\framesubtitle{其他的功能}
\begin{enumerate}
\item 更新软件包 \pkg{pkg1}, \pkg{pkg2}:
\begin{cmdcode}
tlmgr update pkg1 pkg2
\end{cmdcode}
\item 如果更新全部宏包时不想更新 \pkg{pkg1}, \pkg{pkg2}:
\begin{cmdcode}
tlmgr update -self -all -exclude pkg1 -exclude pkg2
\end{cmdcode}
\item 如果更新过程被中断
\begin{cmdcode}
tlmgr update -self -all -reinstall-forcibly-removed
\end{cmdcode}
\end{enumerate}
\end{frame}


\subsection{回滚软件包的版本}

\begin{frame}[fragile]
  \frametitle{回滚软件包的版本 --- 操作: \frameaction{restore}}
  
  \framesubtitle<1>{查看某个软件包的备份} 
  % \begin{onlyenv}<1>
\begin{cmdcode}
tlmgr restore pkg
\end{cmdcode}
\begin{outputcode}
tlmgr restore elegantbook
Available backups for elegantbook: 59053 (2022-04-11 09:12)
\end{outputcode}
  % \end{onlyenv}
  \framesubtitle<2>{将某个软件包回滚为备份}

\begin{uncoverenv}<2>
\begin{cmdcode}
tlmgr restore pkg revision
\end{cmdcode}
\begin{outputcode}
tlmgr restore elegantbook 59053
Do you really want to restore elegantbook to revision 59053 (y/N): y
Restoring elegantbook, 59053 from C:/texlive/2022/tlpkg/backups/elegantbook.r59053.tar.xz
running mktexlsr ...
done running mktexlsr.
running mtxrun --generate ...
done running mtxrun --generate.
tlmgr.pl: package log updated: C:/texlive/2022/texmf-var/web2c/tlmgr.log
tlmgr.pl: command log updated: C:/texlive/2022/texmf-var/web2c/tlmgr-commands.log
\end{outputcode}
\end{uncoverenv}
\end{frame}

\subsection{更多内容}

\begin{frame}
  \frametitle{更多内容}
  如果想了解更多关于 tlmgr 的信息, 欢迎阅读我翻译的 \href{http://mirrors.ctan.org/info/tlmgr-intro-zh-cn/tlmgr-intro-zh-cn.pdf}{tlmgr-intro-zh-cn} 以及 \href{https://www.tug.org/texlive/doc/tlmgr.html}{tlmgr 的官方文档}
\end{frame}




\subsection{检查宏包是否安装}

\begin{frame}[fragile]
  \frametitle{安装外部宏包}
  \framesubtitle{检查一个宏包是否被安装了}
\begin{cmdcode}
tlmgr info pkg
# or 
kpsewhich pkg.sty
\end{cmdcode}
\end{frame}

\begin{frame}[fragile]{安装外部宏包}
\framesubtitle{手动安装宏包}

\begin{enumerate}
  \item 下载或者编写一个宏包文件(cls) 
    \begin{latexcode}
    % mypackage.sty
    \NeedsTeXFormat{LaTeX2e}[2017/04/15]
    \ProvidesPackage{mypackage}[2022/4/20 v1.0 test]
    \newcommand{\mycmd}{Hello \LaTeX}  
    \end{latexcode}
  \item 在 \directory{C:/texlive/texmf-local/tex/latex/local} 文件夹下新建文件夹 \directory{mypackage};
  \item 将 \texttt{mypackage.sty} 放进去;
  \item 命令行运行 \cmd{texhash}.
  \item 调用该宏包:
  \begin{latexcode}
    % main.tex
    \documentclass{article}
    \usepackage{mypackage}
    \begin{document}
      \mycmd
    \end{document}
    \end{latexcode}
\end{enumerate}
\end{frame}


\section{VS Code 的安装与配置}
\subsection{下载并安装 VSCode}
\begin{frame}[fragile]{VS Code 安装 --- Windows 用户}
  \begin{itemize}
    \item 在 VSCode 的官网 \href{https://code.visualstudio.com/}{https://code.visualstudio.com/} 选择 stable 版本进行下载
    \item 安装的时候注意将 ``通过 Code 打开'' 添加到上下文菜单:
  \end{itemize}

  \begin{center}
    \includegraphics[width=8cm]{install-vs.png}
  \end{center}

\end{frame}

\begin{frame}{基础设置与 LaTeX Workshop 安装}
  \begin{enumerate}
    \item 更改语言为中文
    \item 安装 LaTeX Workshop
    \item 配置 LaTeX Workshop 编译链
    \item 更多内容:参见 \href{https://gitee.com/xkwxdyy/CCNUthesis/wikis/\%E5\%B8\%B8\%E8\%A7\%81\%E9\%97\%AE\%E9\%A2\%98FAQ/\%E5\%A6\%82\%E4\%BD\%95\%E5\%AE\%89\%E8\%A3\%85\%E3\%80\%81\%E9\%85\%8D\%E7\%BD\%AE\%E5\%92\%8C\%E4\%BD\%BF\%E7\%94\%A8VScode\#config-LW}{Wiki}. 
  \end{enumerate}
\end{frame}
% !TeX root = main.tex

\section{发行版与模板}
\begin{frame}{发行版与模板}

  \begin{itemize}
    \item \textbf{发行版}:一个 \TeX 发行版是 \TeX 排版引擎、支持排版的文件(基本格式、\LaTeX 宏包、字体等)以
    及一些辅助工具的集合。\\[3ex]
    \item \textbf{模板}:模板本质上是一份实现特定目的、效果的代码,呈现形式一般为 cls(文档类)或者 sty(宏包)。模板具有以下特性:
    \begin{itemize}
      \item 模板维护者需要撰写专门的文档和例子;
      \item 普通用户应该能在阅读模板的使用说明后,较为自主地使用该模板;
      \item 即使是模板,也只为「100\% 使用预定义样式」的使用场景提供便利。
    \end{itemize}
  \end{itemize}
\end{frame}

\begin{frame}[fragile]{模板是如何下载到你本地发行版的}
  模板到你本地发行版里,有几个过程:
  \begin{enumerate}
    \item 将模板上传给 CTAN;
    \item 被发行版收录(如 TeX Live);
    \item 发行版镜像源更新;
    \item 用包管理器安装或升级。
  \end{enumerate}

  \textcolor{iron}{结论:} ElegantLaTeX 的模板在 TeX Live 2019 之后就收录进 TeX Live 里了,也就是说,如果你的发行版在 TeX Live 2019 之后,那么你本地发行版就是有模板的。

  注:可以通过 \lstinline{kpsewhich elegantbook.cls}、\lstinline{texdoc -l elegantbook} 找到模板文件。
\end{frame}

\begin{frame}{Q\&A}
  \textcolor{iron}{问题 1:那为什么还要下载最新版?}
  \begin{itemize}
    \item 模板是有很多版本的,一般来说是通过更新全部宏包来获取最新版模板。
    \item 在需要模板某些新功能的时候,也可以手动下载最新版模板,解压后使用。
  \end{itemize}
  \vspace*{4ex}
  \textcolor{iron}{问题 2:如何获取模板手册?}
  \begin{itemize}
    \item 命令行输入 \lstinline{texdoc elegantbook}。
    \item Github:\href{https://github.com/ElegantLaTeX/ElegantBook/releases}{https://github.com/ElegantLaTeX/ElegantBook/releases}。
  \end{itemize}
\end{frame}

\section{模板使用简介}

\begin{frame}{模板基础设置说明}
  \begin{itemize}
    \item 定理环境
    \item 参考文献
    \item 数学字体
    \item 中文字体
    \begin{itemize}
      \item 字体安装见演示。
      \item 字体命令:\lstinline{fc-cache} 和 \lstinline{fc-list > fontlist.txt}。
    \end{itemize}
  \end{itemize}
\end{frame}


\begin{frame}[standout]
  Thank you!
\end{frame}

\end{document}

