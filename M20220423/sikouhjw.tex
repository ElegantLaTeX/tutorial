% !TeX root = main.tex

\section{发行版与模板}
\begin{frame}{发行版与模板}

  \begin{itemize}
    \item \textbf{发行版}:一个 \TeX 发行版是 \TeX 排版引擎、支持排版的文件(基本格式、\LaTeX 宏包、字体等)以
    及一些辅助工具的集合。\\[3ex]
    \item \textbf{模板}:模板本质上是一份实现特定目的、效果的代码,呈现形式一般为 cls(文档类)或者 sty(宏包)。模板具有以下特性:
    \begin{itemize}
      \item 模板维护者需要撰写专门的文档和例子;
      \item 普通用户应该能在阅读模板的使用说明后,较为自主地使用该模板;
      \item 即使是模板,也只为「100\% 使用预定义样式」的使用场景提供便利。
    \end{itemize}
  \end{itemize}
\end{frame}

\begin{frame}[fragile]{模板是如何下载到你本地发行版的}
  模板到你本地发行版里,有几个过程:
  \begin{enumerate}
    \item 将模板上传给 CTAN;
    \item 被发行版收录(如 TeX Live);
    \item 发行版镜像源更新;
    \item 用包管理器安装或升级。
  \end{enumerate}

  \textcolor{iron}{结论:} ElegantLaTeX 的模板在 TeX Live 2019 之后就收录进 TeX Live 里了,也就是说,如果你的发行版在 TeX Live 2019 之后,那么你本地发行版就是有模板的。

  注:可以通过 \lstinline{kpsewhich elegantbook.cls}、\lstinline{texdoc -l elegantbook} 找到模板文件。
\end{frame}

\begin{frame}{Q\&A}
  \textcolor{iron}{问题 1:那为什么还要下载最新版?}
  \begin{itemize}
    \item 模板是有很多版本的,一般来说是通过更新全部宏包来获取最新版模板。
    \item 在需要模板某些新功能的时候,也可以手动下载最新版模板,解压后使用。
  \end{itemize}
  \vspace*{4ex}
  \textcolor{iron}{问题 2:如何获取模板手册?}
  \begin{itemize}
    \item 命令行输入 \lstinline{texdoc elegantbook}。
    \item Github:\href{https://github.com/ElegantLaTeX/ElegantBook/releases}{https://github.com/ElegantLaTeX/ElegantBook/releases}。
  \end{itemize}
\end{frame}

\section{模板使用简介}

\begin{frame}{模板基础设置说明}
  \begin{itemize}
    \item 定理环境
    \item 参考文献
    \item 数学字体
    \item 中文字体
    \begin{itemize}
      \item 字体安装见演示。
      \item 字体命令:\lstinline{fc-cache} 和 \lstinline{fc-list > fontlist.txt}。
    \end{itemize}
  \end{itemize}
\end{frame}
